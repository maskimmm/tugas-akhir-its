% Judul dokumen
\title{Buku Tugas Akhir ITS}
\author{Musk, Elon Reeve}

% Pengaturan ukuran teks dan bentuk halaman dua sisi
\documentclass[12pt,twoside]{report}

% Pengaturan ukuran halaman dan margin
\usepackage[a4paper,top=30mm,left=30mm,right=20mm,bottom=25mm]{geometry}

% Pengaturan ukuran spasi
\usepackage[singlespacing]{setspace}

% Pengaturan format bahasa
\usepackage[indonesian]{babel}

% Pengaturan detail pada file PDF
\usepackage[pdfauthor={\@author},bookmarksnumbered,pdfborder={0 0 0}]{hyperref}

% Pengaturan jenis karakter
\usepackage[utf8]{inputenc}

% Pengaturan pewarnaan
\usepackage[table,xcdraw]{xcolor}

% Pengaturan kutipan artikel
\usepackage{natbib}

% Package lainnya
\usepackage{changepage}
\usepackage{enumitem}
\usepackage{eso-pic}
\usepackage{etoolbox}
\usepackage{graphicx}
\usepackage{lipsum}
\usepackage{lmodern}
\usepackage{longtable}
\usepackage{tabularx}
\usepackage{wrapfig}

% Definisi untuk "Hati ini sengaja dikosongkan"
\patchcmd{\cleardoublepage}{\hbox{}}{
  \thispagestyle{empty}
  \vspace*{\fill}
  \begin{center}\textit{[Halaman ini sengaja dikosongkan]}\end{center}
  \vfill}{}{}

% Pengaturan penomoran halaman
\usepackage{fancyhdr}
\fancyhf{}
\renewcommand{\headrulewidth}{0pt}
\pagestyle{fancy}
\fancyfoot[LE,RO]{\thepage}
\patchcmd{\chapter}{plain}{fancy}{}{}
\patchcmd{\chapter}{empty}{plain}{}{}

% Pengaturan format judul bab
\usepackage{titlesec}
\titleformat{\chapter}[display]{\bfseries\Large}{BAB \centering\Roman{chapter}}{0ex}{\vspace{0ex}\centering}
\titleformat{\section}{\bfseries\large}{\MakeUppercase{\thesection}}{1ex}{\vspace{1ex}}
\titleformat{\subsection}{\bfseries\large}{\MakeUppercase{\thesubsection}}{1ex}{}
\titleformat{\subsubsection}{\bfseries\large}{\MakeUppercase{\thesubsubsection}}{1ex}{}
\titlespacing{\chapter}{0ex}{0ex}{4ex}
\titlespacing{\section}{0ex}{1ex}{0ex}
\titlespacing{\subsection}{0ex}{0.5ex}{0ex}
\titlespacing{\subsubsection}{0ex}{0.5ex}{0ex}

% Pengaturan format potongan kode
\usepackage{listings}
\definecolor{comment}{RGB}{0,128,0}
\definecolor{string}{RGB}{255,0,0}
\definecolor{keyword}{RGB}{0,0,255}
\lstdefinestyle{codestyle}{
  commentstyle=\color{comment},
  stringstyle=\color{string},
  keywordstyle=\color{keyword},
  basicstyle=\footnotesize\ttfamily,
  numbers=left,
  numberstyle=\tiny,
  numbersep=5pt,
  frame=lines,
  breaklines=true,
  prebreak=\raisebox{0ex}[0ex][0ex]{\ensuremath{\hookleftarrow}},
  showstringspaces=false,
  upquote=true,
  tabsize=2,
}
\lstset{style=codestyle}

% Isi keseluruhan dokumen
\begin{document}

  % Sampul luar Bahasa Indonesia
  \newcommand\covercontents{sampul/konten-id.tex}
  \input{sampul/sampul-luar.tex}

  % Atur ulang penomoran halaman
  \setcounter{page}{1}

  % Sampul dalam Bahasa Indonesia
  \renewcommand\covercontents{sampul/konten-id.tex}
  \input{sampul/sampul-dalam.tex}
  \clearpage
  \cleardoublepage

  % Sampul dalam Bahasa Inggris
  \renewcommand\covercontents{sampul/konten-en.tex}
  \input{sampul/sampul-dalam.tex}
  \cleardoublepage

  % Pengaturan ukuran indentasi paragraf
  \setlength{\parindent}{2em}

  % Pengaturan ukuran spasi paragraf
  \setlength{\parskip}{1ex}

  % Lembar pengesahan
  \begin{center}
	\large
  \textbf{LEMBAR PENGESAHAN}
\end{center}

% Menyembunyikan nomor halaman
\thispagestyle{empty}

\begin{center}
  % Ubah kalimat berikut dengan judul tugas akhir
  \textbf{KALKULASI ENERGI PADA ROKET LUAR ANGKASA BERBASIS \emph{ANTI-GRAVITASI}}
\end{center}

\begingroup
  % Pemilihan font ukuran small
  \small
  
  % \vspace{3ex}

  \begin{center}
    \textbf{TUGAS AKHIR}
    \\Diajukan untuk memenuhi salah satu syarat memperoleh gelar <Sarjana Teknik> pada Program Studi S-1 <Teknik Komputer> Departemen Teknik Komputer Fakultas Teknologi Elektro dan Informatika Cerdas Institut Teknologi Sepuluh Nopember
  \end{center}

  % \vspace{3ex}

  \begin{center}
    % Ubah kalimat berikut dengan nama dan NRP mahasiswa
    Oleh: Elon Reeve Musk 
    \\NRP. 0123 20 4000 0001
  \end{center}

  % \vspace{3ex}

  % \begin{center}
  % Ubah kalimat-kalimat berikut dengan tanggal ujian dan periode wisuda
  %   Tanggal Ujian : 1 Juni 2021\\
  %   Periode Wisuda : September 2021
  % \end{center}

  \begin{center}
    Disetujui oleh Tim Penguji Tugas Akhir:
  \end{center}

  % \vspace{4ex}

  \begingroup
    % Menghilangkan padding
    \setlength{\tabcolsep}{0pt}

    \noindent
    \begin{tabularx}{\textwidth}{X l}
      % Ubah kalimat-kalimat berikut dengan nama dosen pembimbing pertama
      Nikola Tesla, S.T., M.T.          & (Pembimbing I) \\
      NIP: 18560710 194301 1 001        & \\
      & ................................... \\
      &  \\
      &  \\
      % Ubah kalimat-kalimat berikut dengan nama dosen pembimbing kedua
      Wernher von Braun, S.T., M.T.     & (Pembimbing II) \\
      NIP: 18560710 194301 1 001        & \\
      & ................................... \\
      &  \\
      &  \\
      % Ubah kalimat-kalimat berikut dengan nama dosen penguji pertama
      Dr. Galileo Galilei, S.T., M.Sc.  & (Penguji I) \\
      NIP: 18560710 194301 1 001        & \\
      & ................................... \\
      &  \\
      &  \\
      % Ubah kalimat-kalimat berikut dengan nama dosen penguji kedua
      Friedrich Nietzsche, S.T., M.Sc.  & (Penguji II) \\
      NIP: 18560710 194301 1 001        & \\
      & ................................... \\
      &  \\
      &  \\
      % Ubah kalimat-kalimat berikut dengan nama dosen penguji ketiga
      Alan Turing, ST., MT.             & (Penguji III) \\
      NIP: 18560710 194301 1 001        & \\
      & ................................... \\
      &  \\
      &  \\
    \end{tabularx}
  \endgroup

  % \vspace{2ex}

  \begin{center}
    % Ubah kalimat berikut dengan jabatan kepala departemen
    Mengetahui, \\
    Kepala Departemen Teknik Komputer FTEIC - ITS\\

    \vspace{8ex}

    % Ubah kalimat-kalimat berikut dengan nama dan NIP kepala departemen
    \underline{Dr. Supeno Mardi Susiki Nugroho, S.T., M.T.} \\
    NIP. 19700313 199512 1 001
  \end{center}

  \begin{center}
    \textbf{SURABAYA\\Bulan, Tahun}
  \end{center}
\endgroup

  \cleardoublepage
  \begin{center}
	\large
  \textbf{APPROVAL SHEET}
\end{center}

% Menyembunyikan nomor halaman
\thispagestyle{empty}

\begin{center}
  % Ubah kalimat berikut dengan judul tugas akhir
  \textbf{KALKULASI ENERGI PADA ROKET LUAR ANGKASA BERBASIS \emph{ANTI-GRAVITASI}}
\end{center}

\begingroup
  % Pemilihan font ukuran small
  \small
  
  % \vspace{3ex}

  \begin{center}
    \textbf{FINAL PROJECT}
    \\Submitted to fullfill one of the requirements for obtaining Engineering degree at Undergraduate Study Program of Computer Engineering Department of Computer Engineering Faculty of Intelligent Electrical and Informatics Technology Sepuluh Nopember Institute of Technology
  \end{center}

  % \vspace{3ex}

  \begin{center}
    % Ubah kalimat berikut dengan nama dan NRP mahasiswa
    By: Elon Reeve Musk 
    \\NRP. 0123 20 4000 0001
  \end{center}

  % \vspace{3ex}

  % \begin{center}
  % Ubah kalimat-kalimat berikut dengan tanggal ujian dan periode wisuda
  %   Tanggal Ujian : 1 Juni 2021\\
  %   Periode Wisuda : September 2021
  % \end{center}

  \begin{center}
    Approved by Final Project Examiner Team:
  \end{center}

  % \vspace{4ex}

  \begingroup
    % Menghilangkan padding
    \setlength{\tabcolsep}{0pt}

    \noindent
    \begin{tabularx}{\textwidth}{X l}
      % Ubah kalimat-kalimat berikut dengan nama dosen pembimbing pertama
      Nikola Tesla, S.T., M.T.          & (Pembimbing I) \\
      NIP: 18560710 194301 1 001        & \\
      & ................................... \\
      &  \\
      &  \\
      % Ubah kalimat-kalimat berikut dengan nama dosen pembimbing kedua
      Wernher von Braun, S.T., M.T.     & (Pembimbing II) \\
      NIP: 18560710 194301 1 001        & \\
      & ................................... \\
      &  \\
      &  \\
      % Ubah kalimat-kalimat berikut dengan nama dosen penguji pertama
      Dr. Galileo Galilei, S.T., M.Sc.  & (Penguji I) \\
      NIP: 18560710 194301 1 001        & \\
      & ................................... \\
      &  \\
      &  \\
      % Ubah kalimat-kalimat berikut dengan nama dosen penguji kedua
      Friedrich Nietzsche, S.T., M.Sc.  & (Penguji II) \\
      NIP: 18560710 194301 1 001        & \\
      & ................................... \\
      &  \\
      &  \\
      % Ubah kalimat-kalimat berikut dengan nama dosen penguji ketiga
      Alan Turing, ST., MT.             & (Penguji III) \\
      NIP: 18560710 194301 1 001        & \\
      & ................................... \\
      &  \\
      &  \\
    \end{tabularx}
  \endgroup

  % \vspace{2ex}

  \begin{center}
    % Ubah kalimat berikut dengan jabatan kepala departemen
    Acknowledged, \\
    Head of Computer Engineering Department\\

    \vspace{8ex}

    % Ubah kalimat-kalimat berikut dengan nama dan NIP kepala departemen
    \underline{Dr. Supeno Mardi Susiki Nugroho, S.T., M.T.} \\
    NIP. 19700313 199512 1 001
  \end{center}

  \begin{center}
    \textbf{SURABAYA\\Month, Year}
  \end{center}
\endgroup

  \cleardoublepage

  % Pernyataan keaslian
  \input{lainnya/pernyataan-keaslian.tex}
  \cleardoublepage
  \begin{center}
  \large
  \textbf{STATEMENT OF ORIGINALITY}
\end{center}

% Menyembunyikan nomor halaman
\thispagestyle{empty}

\vspace{2ex}

% Ubah paragraf-paragraf berikut sesuai dengan yang ingin diisi pada pernyataan keaslian

\noindent The undersigned below:

\noindent\begin{tabularx}{\textwidth}{X X l}
  & \\
  Name of student / NRP &: Elon Reeve Musk / 0123 20 4000 0001 \\
  Department &: Departemen \\
  Advisor / NIP &: Nama Dosen Pembimbing \\
  & \\
\end{tabularx}

Hereby declared that the Final Project with the title of "" is the result of my own work, is original, and is written by following the rules of scientific writing.

If in future there is a discrepancy with this statement, then I am willing to accept sanctions in accordance with provisions that apply at Sepuluh Nopember Institute of Technology.

\vspace{8ex}

\noindent\begin{tabularx}{\textwidth}{X l}
  % Ubah kalimat berikut sesuai dengan tempat, bulan, dan tahun penulisan
  & Surabaya, Mei 2021\\
  & \\
  Acknowledged & \\
  Advisor & Student\\
  & \\
  & \\
  & \\
  & \\
  & \\
  (Nama Dosen Pembimbing) & (Nama Mahasiswa) \\
  NIP. & NRP. \\
\end{tabularx}
  \cleardoublepage

  % Nomor halaman pembuka dimulai dari sini
  \pagenumbering{roman}

  % Abstrak Bahasa Indonesia
  \input{abstrak/abstrak-id.tex}
  \cleardoublepage

  % Abstrak Bahasa Inggris
  \input{abstrak/abstrak-en.tex}
  \cleardoublepage

  % Kata pengantar
  \input{lainnya/kata-pengantar.tex}
  \cleardoublepage

  % Daftar isi
  \renewcommand*\contentsname{DAFTAR ISI}
  \addcontentsline{toc}{chapter}{\contentsname}
  \tableofcontents
  \cleardoublepage

  % Daftar gambar
  \renewcommand*\listfigurename{DAFTAR GAMBAR}
  \addcontentsline{toc}{chapter}{\listfigurename}
  \listoffigures
  \cleardoublepage

  % Daftar tabel
  \renewcommand*\listtablename{DAFTAR TABEL}
  \addcontentsline{toc}{chapter}{\listtablename}
  \listoftables
  \cleardoublepage

  % Nomor halaman isi dimulai dari sini
  \pagenumbering{arabic}

  % Bab 1 pendahuluan
  \input{bab/1-pendahuluan.tex}
  \cleardoublepage

  % Bab 2 tinjauan pustaka
  \input{bab/2-tinjauan-pustaka.tex}
  \cleardoublepage

  % Bab 3 desain dan implementasi
  \input{bab/3-desain-implementasi.tex}
  \cleardoublepage

  % Bab 4 pengujian dan analisis
  \input{bab/4-pengujian-analisis.tex}
  \cleardoublepage

  % Bab 5 penutup
  \input{bab/5-penutup.tex}
  \cleardoublepage

  % Daftar pustaka
  \renewcommand\bibname{DAFTAR PUSTAKA}
  \addcontentsline{toc}{chapter}{\bibname}
  
  % alternatif
  % \bibliographystyle{apa}
  \bibliographystyle{apacite}
  
  \bibliography{pustaka/pustaka.bib}
  \cleardoublepage

  % Biografi penulis
  \input{lainnya/biografi-penulis.tex}
  \cleardoublepage

\end{document}
