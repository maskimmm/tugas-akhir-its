\chapter{METODOLOGI}
\label{chap:metodologi}

% Ubah bagian-bagian berikut dengan isi dari desain dan implementasi

Penelitian ini dilaksanakan sesuai \lipsum[1][1-5]

% Sepertinya konsepnya sama dengan metode yang digunakan
\section{Deskripsi Sistem}
\label{sec:deskripsisistem}

Sistem akan dibuat dengan \lipsum[1-2]

\section{Peralatan}
\label{sec:peralatan}

Alat yang digunakan yaitu: \lipsum[1]

\subsection{Perangkat}
\label{subsec:perangkat}

Perangkat yang digunakan menggunakan spesifikasi \lipsum[1]

\section{Implementasi Alat}
\label{sec:implementasialat}

% Contoh pembuatan potongan kode
\begin{lstlisting}[
  language=C++,
  caption={Program halo dunia.},
  label={lst:halodunia}
]
#include <iostream>

int main() {
    std::cout << "Halo Dunia!";
    return 0;
}
\end{lstlisting}

\lipsum[2-3]

% Contoh input potongan kode dari file
\lstinputlisting[
  language=Python,
  caption={Program perhitungan bilangan prima.},
  label={lst:bilanganprima}
]{program/bilangan-prima.py}

\lipsum[4]
